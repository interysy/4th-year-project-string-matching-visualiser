
\documentclass[11pt]{article}

\usepackage{times}
\usepackage{fullpage}
\usepackage{hyperref}
\hypersetup{
    colorlinks=true,
    linkcolor=blue,
    filecolor=magenta,
    urlcolor=cyan,
    pdftitle={Overleaf Example},
    pdfpagemode=FullScreen,
}
\providecommand{\tightlist}{%
  \setlength{\itemsep}{0pt}\setlength{\parskip}{0pt}}

\title{ String Matching Algorithms Visualiser}
\author{ Michal Wozniak - 2554268w }

\begin{document}
\maketitle


\section{Status report}

\subsection{Proposal}\label{proposal}

\subsubsection{Motivation}\label{motivation}

There are many visualiser solutions for string matching, but I believe there is room for improvement in areas such as; mobile support, ease of learning and usability. Most solutions are quite barebone, while others are slightly dated and hard to use. My project aims to improve in those areas e.g. adding a visualiser for the O(n) border table creation algorithm, hence helping users (mainly 3rd year computer science students at the University Of Glasgow) learn and understand string matching algorithms - foundational to modern software.

\subsubsection{Aims}\label{aims}

The final deliverable will be a web application with support for KMP, BM and Brute Force approach (hopefully a few more algorithms as well). The app will have the following high-level requirements identified in the requirement-gathering workshop and the prototype survey:
\begin{itemize}
    \tightlist
    \item Pseudocode following along with the animation
    \item Ability to see variable values in real time
    \item Messages reporting what is happening in the animation
    \item Ability to pause, play, move forward and move backwards
    \item Mobile-Friendly Layout
    \item Clear UI
\end{itemize}

Whether my app implements the required features in a way that makes learning easy and the system usable will depend on the evaluation to be completed by participants in the second semester.

\subsection{Progress}\label{progress}

The project is in a good state at the time of writing. Animations for Brute Force, KMP and BM are implemented, including preprocessing animations for BM. All requirements from above are at a minimum in a working state. It is important to note some extra features have also been made:
\begin{itemize}
    \tightlist
\item Ability to filter out preprocessing steps (last occurrence table of BM)
\item Ability to turn toggle smooth animations on/off
\item Automatic animation resizing depending on screen size
\item Last occurrence table for Boyer-Moore being scrollable (layout flexibility)
\end{itemize}

The latest version of the software can be seen at the \href{https://string-search-visualiser-prod.netlify.app/}{following link}. The latest stable version can be seen \href{https://string-search-visualiser-staging.netlify.app/algorithm-visualiser/bruteForce}{here}.

\subsection{Problems and risks}\label{problems-and-risks}

\subsubsection{Problems}\label{problems}

There haven't been too many issues. The biggest blocker I had throughout the first semester was the integration of p5js (used for animations) into Angular, as there is no native support. This means that the "drawing" code can be fragile at times. The other issue that has been a blocker for a while during the first semester is the pseudocode visualiser. Unfortunately, the library (PrismJS) will not let me use my self-defined theme, so the syntax highlighting will need to be implemented natively or scrapped. On top of this, I am still unsure how to display the pseudocode in a clear way (currently it is getting cut off).

\subsubsection{Risks}\label{risks}

The main problem ahead of me is constructing the layout in a way that both pseudocode and animation are clear. My idea at the moment is to make the size of each area resizeable so the user can set the relative sizes to whatever they want.  Another risk I can see is testing for animations. The integration with angular has been a problem in the past, so I am unsure how integrating Mocha (testing framework) will work out. Right now I cannot propose a mitigation, as I haven't done enough research into the framework, but this is something I am looking to do over Christmas.


\subsection{Plan}\label{plan}

\begin{itemize}
\tightlist
\item \textbf{Christmas} Firstly I want to finish smooth animations including colours fading in and out and border table animations for KMP. I also want to handle all responses from the initial prototype survey and create themes (users can pick themes according to preference). If possible I want to start refactoring a little bit to allow for easy integration with new algorithms. If time allows for it I want to research Mocha (p5js testing framework).
    \item \textbf{Week 13} Continue refactoring and final touches.
    \item \textbf{Week 14} Implement 1-2 extra string matching algorithms. Testing
    \item \textbf{Week 15} Testing and Evaluation Plan
    \item \textbf{Week 16} Start writing dissertation and Evaluation Plan
    \item \textbf{Week 17} Release evaluation and continue writing dissertation
    \item \textbf{Week 18-24} Dissertation with evaluation finished up in week 20/21
\end{itemize}


\subsection{Ethics and data}\label{ethics}

My evaluation will likely be a survey combined with focus groups and usability studies. The survey will collect qualitative and quantitative metrics for learning ease, mobile friendliness, user feedback, and areas for improvement. The focus group will be for me to observe how easy people find the software to use and whether there are any stages where people find the system confusing. Lastly, the usability study will be used to determine the usability - was there anything that could have made it better.  The data will be a mix of written responses, quantitative responses and written notes from my observations. To achieve this I will need to have \emph{have verified that the ethics checklist will apply to any evaluation I need to do. I will sign and complete the checklist.}

\end{document}
